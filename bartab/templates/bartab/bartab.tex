

\documentclass[a4paper,oneside,article,11pt,english]{memoir}
\usepackage[margin=1cm]{geometry}
\usepackage[utf8]{inputenc}
\usepackage{microtype}
\usepackage[table]{xcolor}
\usepackage{longtable}
\usepackage{tabu}
\usepackage[breakall,fit]{truncate}
\usepackage{trimclip}
\usepackage{pdflscape}
\usepackage{icomma}
\usepackage{tikz}

\newcommand*\coins[1]{\tikz[baseline=(char.base)]{
			\node[shape=circle,draw,fill=white,inner sep=1pt] at (-0.15, 0) (char) {\phantom{#1}};
			\node[shape=circle,draw,fill=white,inner sep=1pt] at (-0.075, 0) (char) {\phantom{#1}};
            \node[shape=circle,draw,fill=white,inner sep=1pt] (char) {#1};}}
\newcommand\krcoins{\coins{\tiny kr}}

\renewcommand\TruncateMarker{}

\pagestyle{empty}

\begin{document}

\large

\begin{landscape}


	\newpage

\noindent\begin{tabu} to \linewidth{X[l] X[r]}
	{\Large\textbf{{ active_name }} kunder ({{ users|length }})}
	& {\Large Genereret:  } \\
	& {\Large Opgjort til og med vagt: {{ latest_shift|date:"j. F Y" }} }
\end{tabu}

\begin{longtabu} to \linewidth{| X[2, c] | X[4, l] | X[13, l] | X[5, l] | X[5, r, $] | X[48, l] |}
\taburowcolors {white .. gray!50}
\hline
\krcoins & \textbf{Indsat} & \textbf{Navn} & \textbf{Email} & \textbf{Saldo} & \textbf{Køb} \hfill \\ \hline
\endhead

	
&& {{ user.name | latex_trunc }} & {{ user.email | default_if_none:'' | latex_trunc }} & {{ user.balance_str }} & \footnotesize \textbf{KREDITSTOP} \\ \hline
	
\end{longtabu}



% Make sure Pizza list is on a even page
\mbox{~}
\clearpage

\strictpagechecktrue
\checkoddpage
\ifoddpage
\else
\mbox{~}
\clearpage
\fi

\mbox{~}

\end{landscape}

\LARGE
\section*{Pizzabestilling}
\begin{tabu} to \linewidth{| X[1] | X[12] | X[2] |}
\hline
\textbf{Nr.} & \textbf{Navn} & \textbf{Betalt} \\ \hline

& & \\ \hline

\end{tabu}

\end{document}
